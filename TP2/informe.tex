\documentclass[a4paper,titlepage,10pt]{article}

\usepackage[margin=0.6in]{geometry} % margenes
\usepackage[spanish]{babel} % Le indicamos a LaTeX que vamos a escribir en español.
\usepackage[utf8]{inputenc} % Quiero acentos
\usepackage{caratula}
\usepackage{listings}
\usepackage{xcolor}

\titulo{Super Collider}
\fecha{16 / 12 / 2013}
\materia{Teoría de Lenguajes}
\grupo{Grupo 12}
\integrante{Carreiro, Martin}{45/10}{martin301290@gmail.com}
\integrante{Kujawski, Kevin}{459/10}{kevinkuja@gmail.com}
\integrante{Ortiz De Zarate, Juan Manuel}{403/10}{jmanuoz@gmail.com}

\lstset{language=Python,
        morekeywords={as,__init__,MyClass},
        keywordstyle=\color{teal}\bfseries,
        }

\begin{document} % Todo lo que escribamos a partir de aca va a aparecer en el documento.

\maketitle

\section{Introduccíón}


El enunciado plantea la necesidad de generar un parsear para una gramatica y generar sonido a partir de la misma. Para eso usamos la misma gramatica que para el primer TP ya que consideramos correcta y acorde para resolver esta necesidad.

La solución se implemento en Python usando PLY, que usa la tecnica LALR para el analisis sintactico.

Uso:
Primero se debe generar la gramatica con "python main.py". Luego para usar el parser hay dos funciones en el main.py, parsearArchivo(ruta) y parsearCadena(cadena), la cual se le pasa una ruta y una cadena respectivamente.
En el caso del metodo parsearArchivo, el archivo del mismo debera contener una cadena sola, la cual puede tener comentarios (usando //) los cuales se ignoraran y saltos de linea los cuales se concatenaran al cargar el archivo.


Librerias necesarias para PYTHON:
PLY, PYGAMES

Decisiones y aclaraciones:
Por como es el analizador sintactico, por un tema de precedencia, decidimos que los operadores (add, sub, mix, con, div, mul) se van evaluando de derecha a izquierda, y si se quiere evitar esto se deben usar entre llaves. 
Es decir, E OP E debe ir como {E OP E} si se quiere imponer un orden de evaluación.
Ejemplos: 
{{2}+{2;2};{2}} = {4;4;4}
{{2};{2;2}+{2}} = {2;4;4}


Además, aunque hay funciones cuyas especificaciones requieren parametros, en algunos metodos brindamos la posibilidad de no agregarlos y tener valores por defecto.


\section{Código}

\subsection{Lexer}
\lstinputlisting{tokens.py}

\subsection{Parser}
\lstinputlisting{parser.py}

\subsection{Funciones}
\lstinputlisting{funciones.py}

\subsection{Main}
\lstinputlisting{main.py}

\end{document} %Terminé!